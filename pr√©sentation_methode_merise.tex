\hypertarget{merise}{%
\section{Merise}\label{merise}}

\begin{frame}{Qu'es-ce que c'est ?}
\protect\hypertarget{ques-ce-que-cest}{}

Merise est une méthode d'analyse, de conception et de gestion de projet
informatique.

\end{frame}

\begin{frame}{Qui et quand ?}
\protect\hypertarget{qui-et-quand}{}

Dans les années 1970, René Colletti, Arnold Rochfeld et Hubert Tardieu,
après de multiple travaux ont créés la méthode \textbf{Merise}.

\end{frame}

\begin{frame}{Pourquoi ?}
\protect\hypertarget{pourquoi}{}

René Colletti, Arnold Rochfeld et Hubert Tardieu ont décidés de créer
cette méthode afin de pouvoir modéliser les relations existantes entre
plusieurs informations, et de les ordonner entre elles.

\end{frame}

\begin{frame}{Comment ?}
\protect\hypertarget{comment}{}

Les trois créateurs de cette méthode ce sont inspirés des travaux d'un
informaticien britanique, Edgar Frank Codd, sur le modèle relationnel.

\end{frame}