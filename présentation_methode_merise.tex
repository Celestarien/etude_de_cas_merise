\hypertarget{merise}{%
\section{Merise}\label{merise}}

\hypertarget{ques-ce-que-cest}{%
\subsection{Qu'es-ce que c'est ?}\label{ques-ce-que-cest}}

Merise est une méthode d'analyse, de conception et de gestion de projet
informatique. La méthode a aussi connu des tentatives d'adaptation avec
les SGBD (système de gestion de base de données) relationnels. Les
outils du SGBD servent à créer des comptes rendus (reports), des écrans
pour la saisie des informations, importer et exporter les données de et
vers la base de données Ces outils sont utilisés par l'administrateur de
bases de données pour effectuer des sauvegardes, des restaurations de
données, autoriser ou interdire l'accès à certaines informations, et
effectuer des modifications du contenu de la base de données.\\
Les SGBD sont des logiciels complexes et stratégiques, utilisés dans de
très nombreuses applications informatiques, parmi lesquelles le commerce
électronique, les dossiers médicaux, les paiements, les ressources
humaines, la gestion de la relation client et la logistique ainsi que
les blogs et les wikis

\hypertarget{qui-et-quand}{%
\subsection{Qui et quand ?}\label{qui-et-quand}}

Dans les années 1970, René Colletti, Arnold Rochfeld et Hubert Tardieu,
après de multiple travaux ont créés la méthode \textbf{Merise}. En 1960
les informations étaient enregistrées dans des fichiers manipulées par
les logiciels applicatifs2. L'idée des bases de données a été lancée en
1960 dans le cadre du programme Apollo (programme spatial de la NASA).

\hypertarget{pourquoi}{%
\subsection{Pourquoi ?}\label{pourquoi}}

René Colletti, Arnold Rochfeld et Hubert Tardieu ont décidés de créer
cette méthode afin de pouvoir modéliser les relations existantes entre
plusieurs informations, et de les ordonner entre elles.

\hypertarget{comment}{%
\subsection{Comment ?}\label{comment}}

Les trois créateurs de cette méthode ce sont inspirés des travaux d'un
informaticien britanique, Edgar Frank Codd, sur le modèle relationnel.